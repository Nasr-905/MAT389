\chapter{Lecture 3: Series and Sequences}

\begin{definition}
    [infinite Series]
    Suppose we have a sequence:
    \begin{equation}
        z_1, z_2, z_3, \ldots
    \end{equation}
    We can define the partial sum of the sequence as:
    \begin{equation}
        S_n = z_1 + z_2 + z_3 + \ldots + z_n
    \end{equation}
    We say $\sum_{n=1}^{\infty} z_n$ converges and has a sum $S$ if the sequence of partial sums converges to $S$:
    \begin{equation}
        \lim_{n \to \infty} S_n = S
    \end{equation}
    If $\lim_{n \to \infty} S_n$ does not exist, we say the series diverges.
\end{definition}

\begin{corollary}
    [Real and Imaginary Parts of a Series]
    If $\sum_{n=1}^{\infty} z_n$ converges, then the real and imaginary parts of the series also converge.
    \begin{equation}
        \sum_{n=1}^{\infty} z_n = \sum_{n=1}^{\infty} \Re(z_n) + i \sum_{n=1}^{\infty} \Im(z_n)
    \end{equation}
\end{corollary}

\section{Tests for Convergence}
\begin{theorem}
    If $\sum_{n=1}^{\infty} |z_n|$ converges, then so does $|\sum_{n=1}^{\infty} z_n|$ and:
    \[
        \left| \sum_{n=1}^{\infty} z_n \right| \leq \sum_{n=1}^{\infty} |z_n|
    \]
\end{theorem}

\begin{proof}
    Say $z_n = x_n + i y_n$. Then:
    \begin{align*}
        \left| \sum_{n=1}^{\infty} x_n \right| & \leq \sum_{n=1}^{\infty}  \left| x_n \right| \leq \sum_{n=1}^{\infty}  \left| z_n \right| \\
                                               & \text{And}                                                                                \\
        \left| \sum_{n=1}^{\infty} y_n \right| & \leq \sum_{n=1}^{\infty}  \left| y_n \right| \leq \sum_{n=1}^{\infty}  \left| z_n \right|
    \end{align*}
    So if $\sum_{n=1}^{\infty} x_n$ and $\sum_{n=1}^{\infty} y_n$ converge, then $\sum_{n=1}^{\infty} z_n$ converges.
\end{proof}

\begin{table}[htbp]
    \centering
    \begin{tabular}{| m{2cm} | m{5cm} | m{3cm} | m{3cm} |}
        \hline
        \textbf{Test Name}         & \textbf{Description}                                                                                  & \textbf{Conditions for Use}                                               & \textbf{Results}                                                                          \\
        \hline
        Ratio Test                 & \[ \lim_{n \to \infty} \left| \frac{a_{n+1}}{a_n} \right| \]                                          & Applicable when terms are positive and the limit exists.                  & Converges if \( L < 1 \), diverges if \( L > 1 \), inconclusive if \( L = 1 \).           \\
        \hline
        Root Test                  & \[ \lim_{n \to \infty} \sqrt[n]{|a_n|} \]                                                             & Applicable when terms are positive and the limit exists.                  & Converges if \( L < 1 \), diverges if \( L > 1 \), inconclusive if \( L = 1 \).           \\
        \hline
        Integral Test              & Compares a series to an improper. \[ \int_{1}^{\infty} f(x) \, dx \]                                  & Applicable when terms are positive, continuous, and decreasing.           & Converges if the integral converges, diverges if the integral diverges.                   \\
        \hline
        Comparison Test            & Compares a series to a known convergent or divergent series.                                          & Applicable when terms are positive.                                       & Converges if the series being compared to converges.                                      \\
        \hline
        Limit Comparison Test      & Compares the limit of the ratio of terms to a known series. \[ \lim_{n \to \infty} \frac{a_n}{b_n} \] & Applicable when terms are positive and the limit exists.                  & Converges if the limit is finite and the comparison series converges, diverges otherwise. \\
        \hline
        Alternating Series Test    & \[ \sum (-1)^n a_n \quad \text{or}\quad \sum (-1)^{n+1} a_n \]
                                   & When dealing with alternating series                                                                  & Converges if: $a_n > 0$, decreasing, and $\lim_{n\to \infty}a_n = 0     $                                                                                             \\
        \hline
        p-Series Test              & Determines convergence based on the exponent in a series of the form \[ \sum \frac{1}{n^p} \]         & Applicable for series of the form \(\frac{1}{n^p}\).                      & Converges if \( p > 1 \), diverges if \( p \leq 1 \).                                     \\
        \hline
        Geometric Series Test      & Determines convergence for geometric series. \[ \sum ar^n \]                                          & Applicable for series of the form \(ar^n\).                               & Converges if \( |r| < 1 \), diverges if \( |r| \geq 1 \).                                 \\
        \hline
        D'Alembert's Ratio Test    & Similar to the Ratio Test, but specifically for series with factorial terms.                          & Applicable when terms involve factorials.                                 & Converges if the ratio is less than 1, diverges if greater than 1.                        \\
        \hline
        Cauchy's Condensation Test & Determines convergence by condensing the series. \[ \sum a_n \sim \sum 2^n a_{2^n} \]                 & Applicable for series with positive, decreasing terms.                    & Converges if the condensed series converges, diverges if the condensed series diverges.   \\
        \hline
    \end{tabular}
    \caption{Common Tests for Convergence of Series}
    \label{table:convergence_tests}
\end{table}


\begin{example}
    \begin{align}
        \sum_{j=1}^{\infty} j \left( \frac{1 + 2i}{3} \right)^j &
    \end{align}
    We can use the ratio test to determine convergence:
    \begin{align*}
        \sum_{j=1}^{\infty} \left| z_j \right|                 & = \sum_{j=1}^{\infty} j \left| \frac{1 + 2i}{3} \right|^j                                 \\
                                                               & = \sum_{j=1}^{\infty} j \left( \frac{\sqrt{5}}{3} \right)^j                               \\
        \lim_{j \to \infty} \left| \frac{z_{j+1}}{z_j} \right| & = \lim_{j \to \infty} \frac{(j+1)({\frac{\sqrt{5}}{3}})^{j + 1}}{j(\frac{\sqrt{5}}{3})^j} \\
                                                               & = \lim_{j \to \infty} \frac{j+1}{j} \left( \frac{\sqrt{5}}{3} \right)                     \\
                                                               & = \frac{\sqrt{5}}{3} < 1
                                                               & \therefore \text{The series converges}
    \end{align*}
\end{example}

\section{The Exponential Function}
\subsection*{Approach 1}
\begin{definition}
    [Exponential Function]
    If $z = x + iy$, then the exponential function is defined as:
    \begin{equation}
        e^z = e^x \left( \cos(y) + i \sin(y) \right)
    \end{equation}
\end{definition}

\begin{remark}
    [Euler's Formula]
    \begin{align}
        e^{i\theta} & \triangleq \cos(\theta) + i \sin(\theta) \\
    \end{align}
\end{remark}

\subsection*{Properties of the complex Exponential Function}
\begin{table}[htbp]
    \centering
    \begin{tabular}{| m{2.5cm} | m{11.5cm} |}
        \hline
        \textbf{Property} & \textbf{Description}                                                                                                                     \\
        \hline
        Periodicity       & The complex exponential function is periodic with period \( 2\pi i \), \[ e^{z + 2\pi i} = e^z \].                                       \\
        \hline
        Multiplication    & The exponential function satisfies \[ e^{z_1 + z_2} = e^{z_1} e^{z_2} \] for any complex numbers \( z_1 \) and \( z_2 \).                \\
        \hline
        Derivative        & The derivative of the exponential function is \[ \frac{d}{dz} e^z = e^z \].                                                              \\
        \hline
        Inverse           & The inverse of the exponential function is the complex logarithm, \[ \log z \] such that \[ e^{\log z} = z \] for \( z \neq 0 \).        \\
        \hline
        Magnitude         & The magnitude of the exponential function is \[ |e^z| = e^{\Re(z)} \] where \( \Re(z) \) denotes the real part of \( z \).               \\
        \hline
        Argument          & The argument of the exponential function is \[ \arg(e^z) = \Im(z) \mod 2\pi \] where \( \Im(z) \) denotes the imaginary part of \( z \). \\
        \hline
        Conjugate         & The conjugate of the exponential function is \[ \overline{e^z} = e^{\overline{z}} \].                                                    \\
        \hline
    \end{tabular}
    \caption{Properties of the Complex Exponential Function}
    \label{table:complex_exponential_properties}
\end{table}

\subsection*{Approach 2: Taylor Series}

\begin{definition}
    [The Exponential Function]
    The exponential function can be defined as:
    \begin{equation}
        e^z = \sum_{n=0}^{\infty} \frac{z^n}{n!} \qquad \text{for all } z \in \mathbb{C}
    \end{equation}
\end{definition}

\begin{claim}
    [The Taylor Series for the Exponential Function Converges]
    $\sum_{n=0}^{\infty} \frac{z^n}{n!}$ converges for all $z \in \mathbb{C}$.
\end{claim}
\begin{proof}
    HOMEWORK
\end{proof}

\begin{problem}
For $\theta \in \mathbb{R}$
\[
    e^{i\theta} = \sum_{n=0}^{\infty} \frac{(i\theta)^n}{n!} = \cos(\theta) + i \sin(\theta)
\]
\end{problem}

\section{Approach 3: Differential Equations}
\begin{definition}
    [Differential Equation for the Exponential Function]
    The exponential function satisfies the differential equation:
    \begin{equation}
        f(z) = \begin{cases}
            \frac{df}{dz} = f & \text{for all } z \in \mathbb{C} \\
            f(0) = 1
        \end{cases}
    \end{equation}
\end{definition}

\section{The Logarithm Function}
\begin{definition}
    [Logarithm Function]
    The logarithm function is defined as the inverse of the exponential function:
    \begin{equation}
        \log z = \log |z| + i \theta
    \end{equation}
\end{definition}
\begin{remark}
    There will be many solutions to the logarithm function, as the argument is only defined modulo \(2\pi\).
    \[
        \log z = \log |z| + i \left( \arg(z) + 2\pi n \right) \qquad \text{for } n \in \mathbb{Z}
    \]
\end{remark}

\begin{definition}
    [Principal Logarithm]
    The principal branch logarithm is defined as:
    \[
        \text{Log}(z) = \log |z| + i \arg(z) \qquad \text{for } -\pi < \arg(z) \leq \pi
    \]
    \textit{Note: We use a capital L to denote the principal logarithm.}
\end{definition}

\begin{definition}
    [Fixed $\theta_0$ Logarithm Function]
    We can fix the argument of the logarithm function by setting \(\theta_0\) and letting $D = \{ te^{i\theta_0} | t > 0, t \in \mathbb{R} \}$.\\
    We define:
    \[
        \widetilde{\log}_{\theta_0} z = \log |z| + i \left( \widetilde{\arg}(z) + \theta_0 \right) \qquad \text{for } z \in D, \widetilde{\arg}(z) \in [0, 2\pi)
    \]
\end{definition}

\begin{example}
    [Find the Values of $(-1)^i$]
    \begin{align}
        (-1)^i   & = e^{i \log(-1)}                         \\
                 & = e^{i (2n + 1)\pi i}                    \\
                 & = e^{-2n\pi}                             \\
        \log(-1) & = -(2n + 1)\pi i \qquad n \in \mathbb{Z} \\
        (-1)^i   & = e^{2n + 1\pi}
    \end{align}
\end{example}

\section{The Trigonometric Functions}
\begin{definition}
    [Trigonometric Functions]
    For $z \in \mathbb{C}$ trigonometric functions are defined as:
    \begin{align}
        \Re{e^{iz}} & = \cos(z)  = \frac{e^{iz} + e^{-iz}}{2}  \\
        \Im{e^{iz}} & = \sin(z)  = \frac{e^{iz} - e^{-iz}}{2i} \\
        \tan(z)     & = \frac{\sin(z)}{\cos(z)}                \\
    \end{align}
\end{definition}

\begin{lemma}
    \begin{align}
        \begin{cases}
            \cos(z + \alpha) = \cos(z) \\
            \sin(z + \alpha) = \sin(z)
        \end{cases}
    \end{align}
    iff $\alpha = 2\pi n$ for $n \in \mathbb{Z}$.
\end{lemma}

\begin{proof}
    \begin{align}
        e^{i(z + \alpha)} & = e^{iz} e^{i\alpha}                                   \\
                          & = e^{iz} \left( \cos(\alpha) + i \sin(\alpha) \right)  \\
                          & = e^{iz} \left( \cos(2\pi n) + i \sin(2 \pi n) \right) \\
                          & = e^{iz}
    \end{align}
\end{proof}