\chapter{Homework 9}

\begin{example}
    [Fisher, Section 3.5, Problem 2]
    Find a conformal map from
    $D = \{ z \in \mathbb{C} : |\text{Arg}(z)| < \alpha \} \quad \text{for } 0 < \alpha \leq \pi,$
    to the upper half-plane
    $\mathcal{H} = \{ w \in \mathbb{C} : \text{Im}(w) > 0 \}$.

    \hrule
    \vspace{0.5cm}
    % A conformal map $w = f(z)$ preserves angles and maps the wedge-shaped region $D$ to the upper half-plane $\mathcal{H}$. The basic idea is:
    % \begin{itemize}
    %     \item Use a power map $z^p$ to "open up" or "flatten" the wedge.
    %     \item Possibly include a rotation to align the mapped region correctly with $\mathcal{H}$.
    %     \item Ensure the resulting map transforms the boundaries of $D$ to the real axis of $\mathcal{H}$, with the interior of $D$ mapped to the upper half-plane.
    % \end{itemize}

    The wedge $D$ is symmetric about the real axis, with boundaries given by:
    \begin{align*}
        \operatorname{Arg}(z) & = \pm \alpha.
    \end{align*}

    We will use the function:
    \[
        f(z) = \left( z^{\frac{\pi}{2\alpha}} \right) e^{i\frac{\pi}{2}}.
    \]

    Express $z$ in polar form:
    \begin{align*}
        z    & = r e^{i\theta}, \quad \text{so}                                                                            \\
        f(z) & = \left( r^{\frac{\pi}{2\alpha}} e^{i \left( \frac{\pi}{2\alpha} \theta \right)} \right) e^{i\frac{\pi}{2}} \\
             & = r^{\frac{\pi}{2\alpha}} e^{i \left( \frac{\pi}{2\alpha} \theta + \frac{\pi}{2} \right)}.
    \end{align*}

    When $\theta = -\alpha$ (the boundaries of $D$):
    \begin{align*}
        \operatorname{Arg}(f(z)) & = \frac{\pi}{2\alpha} (-\alpha) + \frac{\pi}{2} = -\frac{\pi}{2} + \frac{\pi}{2} = 0.
    \end{align*}

    When $\theta = \alpha$:
    \begin{align*}
        \operatorname{Arg}(f(z)) & = \frac{\pi}{2\alpha} (\alpha) + \frac{\pi}{2} = \frac{\pi}{2} + \frac{\pi}{2} = \pi.
    \end{align*}

    This maps the boundaries of $D$ to the real axis $(\operatorname{Im}(w) = 0)$.

    For $|\operatorname{Arg}(z)| < \alpha$ (the interior of $D$), we have:
    \begin{align*}
        -\alpha < \theta < \alpha \implies -\frac{\pi}{2} + \frac{\pi}{2} < \operatorname{Arg}(f(z)) < \frac{\pi}{2} + \frac{\pi}{2} \implies 0 < \operatorname{Arg}(f(z)) < \pi.
    \end{align*}
    This places the image of the interior of $D$ in the upper half-plane $(\operatorname{Im}(w) > 0)$.

    Thus, the function $f(z) = \left( z^{\frac{\pi}{2\alpha}} \right) e^{i\frac{\pi}{2}}$ is a conformal map from $D$ to $\mathcal{H}$.

    This map transforms the wedge $D$ with angle $2\alpha$ into the upper half-plane, preserving the conformal property.

\end{example}

\begin{example}
    [Fisher, Section 3.5, Problem 3]

    Find a conformal map from $D = \{ x + iy \in \mathbb{C} : |y - 1| < 2 \}$ to the upper half-plane $\mathcal{H} = \{ w \in \mathbb{C} : \operatorname{Im}(w) > 0 \}$.

    \hrule
    \vspace{0.5cm}

    \textbf{1. Understand the Domain $D$} \\

    The domain is given by:
    \begin{align*}
        D = \{ x + iy \in \mathbb{C} : |y - 1| < 2 \}.
    \end{align*}
    This inequality can be rewritten as:
    \begin{align*}
        -2 < y - 1 < 2 \implies -1 < y < 3.
    \end{align*}
    Therefore, $D$ is the horizontal strip in the complex plane where $y \in (-1, 3)$.

    \textbf{2. Shift the Strip Vertically} \\

    To simplify the mapping, we can shift the strip so that it starts at zero. Let's define:
    \begin{align*}
        z' = z + i.
    \end{align*}
    This transformation shifts the imaginary part:
    \begin{align*}
        y' = y + 1.
    \end{align*}
    Now, the strip becomes:
    \begin{align*}
        y' \in (-1 + 1, 3 + 1) \implies y' \in (0, 4).
    \end{align*}
    So the new domain is $y' \in (0, 4)$.

    \textbf{3. Apply the Exponential Function} \\

    The exponential function maps horizontal strips to sectors or the entire complex plane minus a ray. We can use the function:
    \begin{align*}
        w = e^{\frac{\pi}{4} z'}.
    \end{align*}
    Substituting back $z' = z + i$, we get:
    \begin{align*}
        w = e^{\frac{\pi}{4} (z + i)}.
    \end{align*}

    The function:
    \begin{align*}
        f(z) = e^{\frac{\pi}{4} (z + i)}
    \end{align*}
    is a conformal map from the domain $D$ to the upper half-plane $\mathcal{H}$.
\end{example}

\begin{example}
    (Fisher, Section 3.5, Problem 10)

    Find a Schwarz-Christoffel transformation from the upper half-plane $\mathcal{H}$ to $D = \{ z \in \mathbb{C} : 0 < \operatorname{Arg}(z) < \frac{4\pi}{3} \}$.

    \hrule
    \vspace{0.5cm}

    To find the Schwarz-Christoffel transformation from the upper half-plane $\mathcal{H}$ to the sector $D$, we'll proceed step by step:

    \textbf{1. Understand the Target Domain $D$} \\

    The domain $D$ is the sector defined by:
    \begin{align*}
        D = \left\{ z \in \mathbb{C} : 0 < \operatorname{Arg}(z) < \frac{4\pi}{3} \right\}.
    \end{align*}
    This is a sector with an opening angle of $\frac{4\pi}{3}$, bounded by two rays from the origin at angles $0$ and $\frac{4\pi}{3}$.

    \textbf{2. Consider the Mapping Function} \\

    We can consider a mapping of the form:
    \begin{align*}
        w = f(z) = z^\lambda,
    \end{align*}
    where $\lambda$ is a positive real number to be determined. This power function maps the upper half-plane onto a sector in the complex plane.

    \textbf{3. Determine the Appropriate Exponent $\lambda$} \\

    In the upper half-plane $\mathcal{H}$, $z$ has an argument $\theta$ in the range:
    \begin{align*}
        0 < \operatorname{Arg}(z) < \pi.
    \end{align*}
    Under the mapping $w = z^\lambda$, the argument of $w$ becomes:
    \begin{align*}
        \operatorname{Arg}(w) = \lambda \operatorname{Arg}(z).
    \end{align*}
    We want the image of $\mathcal{H}$ under $f(z)$ to be the sector $D$, so we set:
    \begin{align*}
        0 < \operatorname{Arg}(w) = \lambda \operatorname{Arg}(z) < \lambda \pi = \frac{4\pi}{3}.
    \end{align*}
    Solving for $\lambda$:
    \begin{align*}
        \lambda \pi = \frac{4\pi}{3} \implies \lambda = \frac{4}{3}.
    \end{align*}

    \textbf{4. Define the Mapping Function} \\

    With $\lambda = \frac{4}{3}$, the mapping becomes:
    \begin{align*}
        w = f(z) = z^{\frac{4}{3}}.
    \end{align*}

    \textbf{6. Addressing the Branch Cut} \\

    The function $w = z^{\frac{4}{3}}$ is multi-valued, so we need to specify the branch of the logarithm used. We choose the principal branch, where the argument of $z$ satisfies $0 < \operatorname{Arg}(z) < \pi$, corresponding to the upper half-plane.

    The Schwarz-Christoffel transformation from the upper half-plane $\mathcal{H}$ to the sector $D$ is:
    \begin{align*}
        f(z) = z^{\frac{4}{3}}.
    \end{align*}

\end{example}

\begin{example}
    (a) (5 points) Find a conformal map from the infinite strip $\{ z = x + iy : 0 < y < \pi \}$
    to the upper half-plane $\mathcal{H}$.

    (b) (5 points) Using your solution to part (a), and the solution to Q4, find a conformal map from $\{ z = x + iy : 0 < y < \pi \}$ to the semi-infinite strip $\{ \sigma + i\tau : 0 < \sigma < 1, \text{ and } \tau > 0 \}.$

    \hrule
    \vspace{0.5cm}

    \textbf{Part (a)} \\

    To find a conformal map from the infinite strip $\{ z = x + iy : 0 < y < \pi \}$ to the upper half-plane $\mathcal{H}$, we can use the exponential function:

    \begin{align*}
        w = f(z) = e^{z}.
    \end{align*}

    \textbf{Verification of the Mapping} \\

    Let $z = x + iy$, then:

    \begin{align*}
        w & = e^{x + iy}                 \\
          & = e^{x} e^{i y}              \\
          & = e^{x} (\cos y + i \sin y).
    \end{align*}

    Since $0 < y < \pi$, we have $\sin y > 0$. Therefore, the imaginary part of $w$ is:

    \begin{align*}
        \operatorname{Im}(w) = e^{x} \sin y > 0.
    \end{align*}

    This means that $w$ lies in the upper half-plane $\mathcal{H}$. The mapping $w = e^{z}$ is conformal because the exponential function is analytic everywhere and its derivative is never zero.
    \textbf{Part (b)} \\

    Using the solution from part (a) and the solution to Q4, we aim to find a conformal map from $\{ z = x + iy : 0 < y < \pi \}$ to the semi-infinite strip $\{ \sigma + i\tau : 0 < \sigma < 1, \tau > 0 \}$.

    \textbf{Step 1: Map the Strip to the Upper Half-Plane} \\

    From part (a), we have the mapping:

    \begin{align*}
        w = e^{z},
    \end{align*}

    which maps the strip $\{ z = x + iy : 0 < y < \pi \}$ to the upper half-plane $\mathcal{H}$.

    \textbf{Step 2: Map the Upper Half-Plane to the Semi-Infinite Strip} \\

    We need a conformal map from $\mathcal{H}$ to the semi-infinite strip $\{ \sigma + i\tau : 0 < \sigma < 1, \tau > 0 \}$. One such map, inspired by the solution to Q4, is:

    \begin{align*}
        s = \frac{1}{\pi} \ln w.
    \end{align*}

    \textbf{Verification of the Mapping} \\

    Let $w = u + i v$, where $v > 0$ (since $w \in \mathcal{H}$). Then:

    \begin{align*}
        s & = \frac{1}{\pi} \ln w                                \\
          & = \frac{1}{\pi} (\ln |w| + i \operatorname{Arg}(w)).
    \end{align*}

    Since $w$ is in the upper half-plane, $\operatorname{Arg}(w) \in (0, \pi)$. Therefore:

    \begin{align*}
        \operatorname{Im}(s) = \frac{1}{\pi} \operatorname{Arg}(w) \in \left( 0, 1 \right).
    \end{align*}

    However, the real part $\operatorname{Re}(s) = \frac{1}{\pi} \ln |w|$ varies over $\mathbb{R}$.

    \textbf{Step 3: Map the Infinite Strip to the Semi-Infinite Strip} \\

    To restrict the real part to $(0,1)$, we can use the transformation:

    \begin{align*}
        f(w) = \frac{1}{2} + \frac{1}{2} \tanh\left( \frac{\pi}{2} s \right).
    \end{align*}

    Substituting $s = \frac{1}{\pi} \ln w$, we get:

    \begin{align*}
        f(w) & = \frac{1}{2} + \frac{1}{2} \tanh\left( \frac{\pi}{2} \left( \frac{1}{\pi} \ln w \right) \right) \\
             & = \frac{1}{2} + \frac{1}{2} \tanh\left( \frac{1}{2} \ln w \right).
    \end{align*}

    Simplify the expression:

    \begin{align*}
        f(w) & = \frac{1}{2} + \frac{1}{2} \cdot \frac{w^{1/2} - w^{-1/2}}{w^{1/2} + w^{-1/2}}                       \\
             & = \frac{1}{2} + \frac{1}{2} \cdot \frac{\sqrt{w} - \frac{1}{\sqrt{w}}}{\sqrt{w} + \frac{1}{\sqrt{w}}} \\
             & = \frac{1}{2} + \frac{1}{2} \cdot \frac{w - 1}{w + 1}.
    \end{align*}

    Therefore, the mapping from $w$ to the semi-infinite strip is:

    \begin{align*}
        f(w) = \frac{1}{2} + \frac{1}{2} \cdot \frac{w - 1}{w + 1}.
    \end{align*}

    \textbf{Composite Mapping} \\

    Combining the mappings from Step 1 and Step 3, we have:

    \begin{align*}
        \text{First, } w   & = e^{z},                                                       \\
        \text{Then, } f(w) & = \frac{1}{2} + \frac{1}{2} \cdot \frac{e^{z} - 1}{e^{z} + 1}.
    \end{align*}

    Simplify the composite mapping:

    \begin{align*}
        f(z) & = \frac{1}{2} + \frac{1}{2} \cdot \frac{e^{z} - 1}{e^{z} + 1} \\
             & = \frac{1}{2} + \frac{1}{2} \tanh\left( \frac{z}{2} \right).
    \end{align*}

    \textbf{Conclusion for Part (b)} \\

    The conformal map from $\{ z = x + iy : 0 < y < \pi \}$ to the semi-infinite strip $\{ \sigma + i\tau : 0 < \sigma < 1, \tau > 0 \}$ is:

    \begin{align*}
        f(z) = \frac{1}{2} + \frac{1}{2} \tanh\left( \frac{z}{2} \right).
    \end{align*}

\end{example}

\begin{example}
    Find a Schwarz-Christoffel transformation mapping the upper half-plane $\mathcal{H}$ onto the given domain $D$.

    \textbf{Solution} \\

    To find the Schwarz-Christoffel transformation for the modified domain $D$, we proceed as follows:

    \textbf{1. Analyze the Geometry of the Domain} \\

    The domain $D$ now consists of:
    \begin{itemize}
        \item A vertical line segment from $0$ to $i a$ along the positive imaginary axis.
        \item A vertical line segment from $0$ to $-i \infty$ along the negative imaginary axis.
        \item A horizontal line segment from $0$ to $-\infty$ along the negative real axis.
    \end{itemize}

    \textbf{2. Identify the Vertices and Angles} \\

    We identify the vertices of $D$ and their corresponding interior angles:
    \begin{itemize}
        \item At $z = 0$: This is a vertex where three edges meet. The interior angle is $\frac{3\pi}{2}$.
        \item At $z = i a$: The angle here is $\pi$ (since the path continues straight upward).
        \item At infinity along the negative real axis and negative imaginary axis.
    \end{itemize}

    \textbf{3. Calculate the Exponents $\beta_k$} \\

    For each vertex, we calculate:
    \begin{align*}
        \beta_k = \frac{\text{Interior angle at vertex}}{\pi} - 1.
    \end{align*}

    \begin{itemize}
        \item At $z = 0$:
              \begin{align*}
                  \beta_1 = \frac{\frac{3\pi}{2}}{\pi} - 1 = \frac{3}{2} - 1 = \frac{1}{2}.
              \end{align*}
        \item At $z = i a$:
              \begin{align*}
                  \beta_2 = \frac{\pi}{\pi} - 1 = 0.
              \end{align*}
    \end{itemize}

    \textbf{4. Write the Schwarz-Christoffel Transformation} \\

    The general form of the Schwarz-Christoffel transformation is:
    \begin{align*}
        f(z) = C \int \prod_{k} (z - z_k)^{\beta_k} \, dz + C_0,
    \end{align*}
    where $C$ and $C_0$ are constants, $z_k$ are the pre-images of the vertices, and $\beta_k$ are the exponents calculated above.

    Substituting the values:
    \begin{align*}
        f(z) = C \int (z - 0)^{\beta_1} (z - i a)^{\beta_2} \, dz + C_0.
    \end{align*}

    Since $\beta_2 = 0$, $(z - i a)^{\beta_2} = 1$. Therefore, the transformation simplifies to:
    \begin{align*}
        f(z) = C \int z^{\frac{1}{2}} \, dz + C_0.
    \end{align*}

    \textbf{5. Evaluate the Integral} \\

    Integrate $z^{\frac{1}{2}}$:
    \begin{align*}
        \int z^{\frac{1}{2}} \, dz = \frac{2}{3} z^{\frac{3}{2}} + \text{constant}.
    \end{align*}

    Therefore, the Schwarz-Christoffel transformation becomes:
    \begin{align*}
        f(z) = C \left( \frac{2}{3} z^{\frac{3}{2}} \right) + C_0.
    \end{align*}

    \textbf{6. Determine the Constants} \\

    The constants $C$ and $C_0$ can be determined based on boundary conditions or normalization requirements specific to the problem.

    \textbf{7. Conclusion} \\

    With the additional line segment from $0$ to $-i \infty$ along the negative imaginary axis, the Schwarz-Christoffel transformation mapping the upper half-plane $\mathcal{H}$ onto the domain $D$ is:
    \begin{align*}
        f(z) = C \left( \frac{2}{3} z^{\frac{3}{2}} \right) + C_0.
    \end{align*}
\end{example}

\begin{example}
    To find the Schwarz-Christoffel transformation mapping the upper half-plane $\mathcal{H}$ onto the domain $D$, we proceed as follows:

    \textbf{1. Analyze the Geometry of the Domain $D$} \\

    The domain $D$ is composed of:

    \begin{itemize}
        \item Quadrant 1: $0 < \theta < \dfrac{\pi}{2}$
        \item Quadrant 2: $\dfrac{\pi}{2} < \theta < \pi$
        \item Quadrant 4: $-\dfrac{\pi}{2} < \theta < 0$
        \item An additional line segment from the origin in the direction $\theta = \dfrac{\pi}{4}$.
    \end{itemize}

    \textbf{2. Identify the Vertices and Angles} \\

    We need to identify the vertices of the polygonal domain $D$ and their corresponding interior angles.

    The vertices are:

    \begin{itemize}
        \item At $z = 0$: This point is a vertex where three edges meet—the rays at angles $-\dfrac{\pi}{2}$, $0$, and $\dfrac{\pi}{2}$, with a slit along $\theta = \dfrac{\pi}{4}$. The interior angle at $z = 0$ is calculated as:
              \begin{align*}
                  \text{Interior angle at } z = 0 & = 2\pi - \left( \dfrac{\pi}{2} + \dfrac{\pi}{4} \right) = \dfrac{5\pi}{4}.
              \end{align*}
        \item At infinity along $\theta = -\dfrac{\pi}{2}$ (negative imaginary axis): The interior angle is $\dfrac{\pi}{2}$.
        \item At infinity along $\theta = \pi$ (negative real axis): The interior angle is $\pi$.
        \item At infinity along $\theta = \dfrac{\pi}{4}$ (the slit): The interior angle is $0$ (since the slit introduces a branch cut).
    \end{itemize}

    \textbf{3. Calculate the Exponents $\beta_k$} \\

    For each vertex, we calculate:
    \begin{align*}
        \beta_k = \dfrac{\text{Interior angle at vertex}}{\pi} - 1.
    \end{align*}
    \begin{itemize}
        \item At $z = 0$:
              \begin{align*}
                  \beta_1 = \dfrac{\dfrac{5\pi}{4}}{\pi} - 1 = \dfrac{5}{4} - 1 = \dfrac{1}{4}.
              \end{align*}
        \item At $\theta = -\dfrac{\pi}{2}$ (infinity along negative imaginary axis):
              \begin{align*}
                  \beta_2 = \dfrac{\dfrac{\pi}{2}}{\pi} - 1 = \dfrac{1}{2} - 1 = -\dfrac{1}{2}.
              \end{align*}
        \item At $\theta = \pi$ (infinity along negative real axis):
              \begin{align*}
                  \beta_3 = \dfrac{\pi}{\pi} - 1 = 1 - 1 = 0.
              \end{align*}
        \item At $\theta = \dfrac{\pi}{4}$ (infinity along the slit):
              \begin{align*}
                  \beta_4 = \dfrac{0}{\pi} - 1 = -1.
              \end{align*}
    \end{itemize}

    \textbf{4. Write the Schwarz-Christoffel Transformation} \\

    The general Schwarz-Christoffel transformation is:
    \begin{align*}
        f(z) = C \int \prod_{k} (z - z_k)^{\beta_k} \, dz + C_0.
    \end{align*}

    Since two of the vertices are at infinity, we adjust the mapping accordingly. Let's assign finite pre-images to the vertices:
    \begin{itemize}
        \item $z = -1$: Pre-image of $\theta = \pi$ (negative real axis).
        \item $z = 0$: Vertex at $z = 0$.
        \item $z = 1$: Pre-image of $\theta = \dfrac{\pi}{4}$ (the slit).
        \item $z = \infty$: Corresponds to $\theta = -\dfrac{\pi}{2}$ (negative imaginary axis).
    \end{itemize}

    The mapping becomes:
    \begin{align*}
        f(z) = C \int (z + 1)^{\beta_3} z^{\beta_1} (z - 1)^{\beta_4} \, dz + C_0.
    \end{align*}

    Substituting the exponents:
    \begin{align*}
        f(z) & = C \int (z + 1)^{0} \, z^{\frac{1}{4}} \, (z - 1)^{-1} \, dz + C_0 \\
             & = C \int z^{\frac{1}{4}} (z - 1)^{-1} \, dz + C_0.
    \end{align*}

    \textbf{5. Evaluate the Integral} \\

    The integral:
    \begin{align*}
        I = \int z^{\frac{1}{4}} (z - 1)^{-1} \, dz
    \end{align*}
    can be evaluated using substitution or special functions, but it may involve complex analysis techniques. For the purpose of this problem, we can leave it in integral form or express it using known functions.

    Therefore, the Schwarz-Christoffel transformation is:
    \begin{align*}
        f(z) = C \int z^{\frac{1}{4}} (z - 1)^{-1} \, dz + C_0,
    \end{align*}
    where $C$ and $C_0$ are constants determined by boundary conditions.
\end{example}