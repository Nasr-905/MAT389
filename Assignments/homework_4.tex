\chapter{Homework 4}

\begin{example}
    [Fisher, Section 2.4, Problem 10]

    find the power series expansion of $f(z) = e^z$ about the point $z_0 = \pi i$. What is the largest disc on which this series is valid?

    \hrule
    \vspace{0.5cm}

    We know that a taylor series is given by

    \begin{equation}
        f(z) = \sum_{n=0}^{\infty} \frac{f^{(n)}(z_0)}{n!} (z - z_0)^n
    \end{equation}
    for $f(z) = e^z$, we know that $f^{(n)}(z) = e^z$ for all $n \in \mathbb{N}$. Therefore, the taylor series expansion of $f(z)$ about $z_0 = \pi i$ is

    \begin{equation}
        e^z = \sum_{n=0}^{\infty} \frac{e^{\pi i}}{n!} (z - \pi i)^n
    \end{equation}

    The largest disk on which this series is valid can be found by using the ratio test. The ratio test states that a series converges if the limit of the ratio of the $(n+1)$th term to the $n$th term is less than 1. That is

    \begin{equation}
        \lim_{n \to \infty} \left| \frac{a_{n+1}}{a_n} \right| < 1
    \end{equation}

    for the series $\sum a_n$. In this case, we have

    \begin{equation}
        \lim_{n \to \infty} \left| \frac{e^{\pi i}}{(n+1)!} (z - \pi i)^{n+1} \cdot \frac{n!}{e^{\pi i}} (z - \pi i)^n \right|
    \end{equation}

    Simplifying, we get

    \begin{equation}
        \lim_{n \to \infty} \left| \frac{z - \pi i}{n+1} \right| = 0
    \end{equation}

    Therefore, $R = \frac{1}{L} = \infty$. This means that the series converges for all $z \in \mathbb{C}$.

\end{example}

\begin{example}
    [Fisher, Section 2.4, Problem 12]
    Find the power series expansion of $f(z) = \frac{z^2}{1-z}$ about $z_0 = 0$. What is the largest disc on which this series is valid?

    \hrule
    \vspace{0.5cm}

    We know that $\frac{1}{1-z}$ has a taylor series expansion about $z_0 = 0$ given by
    \begin{align*}
        \frac{1}{1-z} & = \sum_{n=0}^{\infty} z^n
    \end{align*}
    So, the taylor series expansion of $f(z)$ about $z_0 = 0$ is
    \begin{align*}
        f(z) & = z^2 \sum_{n=0}^{\infty} z^n \\
             & = \sum_{n=0}^{\infty} z^{n+2}
             & = \sum_{n=2}^{\infty} z^n
    \end{align*}

    The radius of convergence of this series is inherited from the radius of convergence of the series for $\frac{1}{1-z}$, which is $R = 1$.
\end{example}

\begin{example}
    [Fisher, Section 2.4, Problem 18]

    Using the (consequence of) the Cauchy Integral Formula, prove the Cauchy estimates:
    \[
        \left| f^{(n)}(z_0) \right| \leq \frac{n!}{r^n} \max_{\{|z - z_0| = r\}} |f(z)|, \quad n = 0, 1, 2, \dots,
    \]
    whenever \( f \) is analytic on a domain containing the set \( \{ |z - z_0| < r \} \).

    \hrule
    \vspace{0.5cm}

    If $f(z)$ is analytic on a domain containing the set $\{|z - z_0| < r\}$, then there exists a power series expansion of $f(z)$ about $z_0$ given by
    \begin{equation}
        f(z) = \sum_{n=0}^{\infty} a_n (z - z_0)^n
    \end{equation}

    We know that the \(n\)th derivative of \(f(z)\) is given by

    \begin{align*}
        f^{(n)}(z)   & = \sum_{k=n}^{\infty} a_k \frac{k!}{(k-n)!} (z - z_0)^{k-n} \\
        f^{(n)}(z_0) & = a_n n!
    \end{align*}

    The coefficient \(a_n\) are given by the following line integral where $\gamma$ is a simple, closed, positively oriented curve in $\{|z - z_0| < r\}$:

    \begin{align*}
        a_n            & = \frac{1}{2\pi i}\int_{\gamma} \frac{f(z)}{(z - z_0)^{n+1}} dz                                          \\
        a_n n!         & = \frac{n!}{2\pi i}\int_{\gamma} \frac{f(z)}{(z - z_0)^{n+1}} dz                                         \\
                       & \leq \frac{n!}{2\pi} \max_{\{|z - z_0| = r\}} \left|\int_{\gamma} \frac{f(z)}{(z - z_0)^{n+1}} dz\right| \\
                       & \leq \frac{n!}{2\pi} 2\pi r\max_{\{|z - z_0| = r\}} |\frac{f(z)}{r^{n+1}}|                               \\
                       & \leq n! \max_{\{|z - z_0| = r\}} \frac{|f(z)|}{r^{n}}                                                    \\
        |f^{(n)}(z_0)| & \leq \frac{n!}{r^{n}} \max_{\{|z - z_0| = r\}} |f(z)|
    \end{align*}

    As required.
\end{example}

\begin{example}
    [Fisher, Section 2.4, Problem 20]

    Suppose that $f(z)$ is an entire function and $\Re f(z) \leq c$ for all $z$. Show that $f$ is constant. (Hint: Consider $e^{f(z)}$.)

    \hrule
    \vspace{0.5cm}

    if $f(z)$ is an entire function, and $\Re f(z) \leq c$ for all $z$, then:
    \begin{align*}
        \Re f(z)     & \leq c                \\
        e^{\Re f(z)} & \leq e^c              \\
        g(z)         & = |e^{f(z)}| \leq e^c
    \end{align*}
    So by Liouville's theorem, $g(z) = C$ is constant and by extension:
    \begin{align*}
        e^{f(z)} & = C      \\
        f(z)     & = \log C
    \end{align*}
    $f(z)$ is also constant.
\end{example}

\begin{example}
    [Fisher, Section 2.4, Problem 21]
    Suppose that \( f(z) \) is an entire function and that there are constants \( A \), \( R_0 > 0 \) and \( m \in \mathbb{Z}_{>0} \) so that
    \[
        |f(z)| \leq A |z|^m \quad \text{for all } |z| \geq R_0.
    \]
    Show that \( f \) is a polynomial of degree at most \( m \). (Hint: Use Problem 3, above, for \( n > m \), and let \( r \to \infty \).)

    \hrule
    \vspace{0.5cm}

    The cauchy estimates state that for an entire function \( f(z) \) with a power series expansion about \( z_0 \) given by
    \[
        f(z) = \sum_{n=0}^{\infty} a_n (z - z_0)^n
    \]

    we have, for $z_0 = 0$:

    \begin{align*}
        |f^{(n)}(z_0)| & \leq \frac{n!}{r^n} \max_{\{|z - z_0| = r\}} |f(z)| \\
        |f^{(n)}(z_0)| & \leq \frac{n!}{r^n} \max_{\{|z| = r\}} A|z|^m       \\
        |f^{(n)}(z_0)| & \leq n! A r^{m-n}
    \end{align*}

    If \( f(z) \) is an entire function, then it must be defined for $r \to \infty$.

    \begin{align*}
        \lim_{r \to \infty} |f^{(n)}(z_0)| & \leq \lim_{r \to \infty} n! A r^{m-n} = \begin{cases}
                                                                                         0      & \text{if } n > m \\
                                                                                         n!A    & \text{if } n = m \\
                                                                                         \infty & \text{if } n < m
                                                                                     \end{cases}
    \end{align*}

    Because the limit of the $n^{th}$ derivative of \(f(z)\) as \(r \to \infty\) is 0 for \(n > m\), this implies that the power series expansion of \(f(z)\) is a polynomial of degree at most \(m\).
\end{example}

\begin{example}
    [Fisher, Section 2.4, Problem 22]

    Let \( D \) be a simply connected domain and \( f \) an analytic function on \( D \) that has no zeroes in \( D \). Let \( \gamma \neq 0 \) be a non-zero complex number. Show that there is an analytic function \( g \) on \( D \) such that \( f = g^{\gamma} \).

    \hrule
    \vspace{0.5cm}

    \begin{align*}
        f(z)                     & = g^{\gamma}                   \\
        \log f(z)                & = \gamma \log g                \\
        \frac{\log f(z)}{\gamma} & = \log g                       \\
        g(z)                     & = e^{\frac{\log f(z)}{\gamma}}
    \end{align*}
    Because $f(z)$ has no zeroes in $D$, we can define a branch of the logarithm, $\log f(z)$, such that it is analytic in $D$. And since $\gamma \neq 0$, $\frac{\log f(z)}{\gamma}$ is also analytic in $D$. Therefore, $g(z)$ is analytic in $D$.
\end{example}

\begin{example}
    [Fisher, Section 2.4, Problem 23]

    Locate the isolated singularities of \( f(z) = \frac{z^2}{\sin(z)} \). Identify each singularity as a removable singularity, a pole, or an essential singularity. If the singularity is removable, give the value of the function at the point.

    \hrule
    \vspace{0.5cm}

    We know $\sin(z)$ has zeros at $z = n\pi$ for $n \in \mathbb{Z}$. Therefore, $f(z)$ has singularities at $z = n\pi$ for $n \in \mathbb{Z}$. Near $z = n\pi$, $\sin(n \pi) \approx (-1)^n(z - n\pi)$. We can classify these singularities by examining the limit of $f(z)$ as $z \to n\pi$.

    \begin{align*}
        f(z) & = \frac{z^2}{\sin(z)}          \\
             & = \frac{z^2}{(-1)^n(z - n\pi)} \\
             & = \frac{z^2}{(-1)^n(z - n\pi)} \\
    \end{align*}
    at $z = 0$, $f(z)$ has a removable singularity, because
    \begin{align*}
        f(z) & = \frac{z^2}{(-1)^n(z - n\pi)} \\
             & = \frac{z^2}{(-1)^0(z - 0)}    \\
             & = z
    \end{align*}
    at $z = n\pi$, $f(z)$ has a pole of order 1, because
    \begin{align*}
        f(z) & = \frac{z^2}{(-1)^n(z - n\pi)} \\
             & = \frac{z^2}{(-1)^n(z - n\pi)} \\
    \end{align*}
\end{example}

\begin{example}
    [Fisher, Section 2.5, Problem 4]
    Locate the isolated singularities of \( f(z) = \pi \cot(\pi z) \). Identify each singularity as a removable singularity, a pole, or an essential singularity. If the singularity is removable, give the value of the function at the point.

    \hrule
    \vspace{0.5cm}

    \begin{align*}
        f(z) & = \pi \cot(\pi z)                     \\
             & = \pi \frac{\cos(\pi z)}{\sin(\pi z)} \\
             & \approx \pi \frac{(-1)^n}{\pi(z - n)} \\
    \end{align*}
    The above approximation is valid near $z = n$ for $n \in \mathbb{Z}$. We can classify these singularities by examining the limit of $f(z)$ as $z \to n$.

    \begin{align*}
        f(z)                & = \pi \frac{(-1)^n}{\pi(z - n)}                               \\
                            & = \frac{(-1)^n}{z - n}                                        \\
        \lim_{z \to n} f(z) & = \frac{(-1)^n}{0} = \infty \quad \text{On both sides of $n$}
    \end{align*}
    Therefore, $f(z)$ has poles of order 1 at $z = n$ for $n \in \mathbb{Z}$.
\end{example}

\begin{example}
    [Fisher, Section 2.5, Problem 6]

    Locate the isolated singularities of \( f(z) = \frac{e^z - 1}{e^{2z} - 1} \). Identify each singularity as a removable singularity, a pole, or an essential singularity. If the singularity is removable, give the value of the function at the point.

    \hrule
    \vspace{0.5cm}

    We can start by finding the zeros of the denominator, $e^{2z} - 1 = 0$:

    \begin{align*}
        e^{2z} - 1 & = 0                                           \\
        e^{2z}     & = 1                                           \\
        2z         & = 2\pi i n \quad \text{for } n \in \mathbb{Z} \\
        z          & = \pi i n \quad \text{for } n \in \mathbb{Z}
    \end{align*}

    To examine the singularities of $f(z)$, we can simplify the function:
    \begin{align*}
        f(z)   & = \frac{e^z - 1}{e^{2z} - 1}         \\
               & = \frac{e^z - 1}{(e^z - 1)(e^z + 1)} \\
               & = \frac{1}{e^z + 1}                  \\
        f(z_0) & = \frac{1}{e^{z_0} + 1}              \\
               & = \frac{1}{e^{\pi i n} + 1}          \\
               & = \frac{1}{1 + 1} = \frac{1}{2}
    \end{align*}
    Therefore, $f(z)$ has removable singularities at $z = \pi i n$ for $n \in \mathbb{Z}$ equal to $\frac{1}{2}$.
\end{example}

\begin{example}
    [Fisher, Section 2.5, Problem 16]

    Suppose \( f \) and \( g \) are analytic in \( |z - z_0| < R \) and \( f \) has a zero of order \( m \geq 1 \) at \( z_0 \), while \( g \) has a zero of order \( l \leq m \) at \( z_0 \). Show that \( \frac{f}{g} \) has a removable singularity at \( z_0 \).

    \hrule
    \vspace{0.5cm}

    Since $f(z)$ and $g(z)$ are analytic in $|z - z_0| < R$, we can write their zeroes around $z_0$ as:

    \begin{align*}
        f(z) = (z - z_0)^m f_1(z) \\
        g(z) = (z - z_0)^l g_1(z)
    \end{align*}

    where $f_1(z)$ and $g_1(z)$ are analytic at $z_0$ and $f_1(z_0) \neq 0$ and $g_1(z_0) \neq 0$. We can then write $\frac{f}{g}$ as:

    \begin{align*}
        \frac{f}{g}      & = \frac{(z - z_0)^m f_1(z)}{(z - z_0)^l g_1(z)} \\
                         & = (z - z_0)^{m-l} \frac{f_1(z)}{g_1(z)}         \\
        \frac{f}{g}(z_0) & = \begin{cases}
                                 0                         & \text{if } m > l \\
                                 \frac{f_1(z_0)}{g_1(z_0)} & \text{if } m = l
                             \end{cases}
    \end{align*}

    Therefore, $\frac{f}{g}$ has a removable singularity at $z_0$.
\end{example}